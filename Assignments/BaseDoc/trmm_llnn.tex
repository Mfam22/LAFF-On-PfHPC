\chapter{$ B := L B $ --  \large Team: Robert van de Geijn}



\section{Operation}

Consider the operation
\[
B := L B 
\]
where $ L $ is a $ m \times m $ lower triangular matrix and $ B $ is a $ m \times n $ matrix.
This is a special case of triangular 
matrix-matrix multiplication, 
with the {\sc l}ower triangular matrix on the {\sc left}, 
and the triangular matrix is {\sc n}ot transposed.
We will refer to this operation
as {\sc Trmm\_llnn} where the {\sc llnn} stands for
\underline{l}eft
\underline{l}ower
\underline{n}o-transpose
\underline{n}onunit diagonal.
The nonunit diagonal means we will use the entries of the matrix that are stored on the diagonal.

\section{Precondition and postcondition}

In the precondition 
\[
B = \widehat B
\]
$ \widehat B $ denotes the original contents of $ B $.
This allows us to express the state upon completion, the postcondition, as
\[
B = L \widehat B.
\]

\section{Partitioned Matrix Expressions and loop invariants}

There are two PMEs for this operation.

\subsection{PME 1}

To derive the second PME, partition
\[
L \rightarrow 
\FlaTwoByTwo
	{ L_{TL}}{ 0 }
	{ L_{BL}}{ L_{BR}},
	\quad \mbox{and} \quad
B \rightarrow 
\FlaTwoByOne
	{ B_T }
	{ B_B }.
\]
Substituting these into the postcondition
yields
\[
\FlaTwoByOne
{ B_T }
{ B_B }
=
\FlaTwoByTwo
{ L_{TL}}{ 0 }
{ L_{BL}}{ L_{BR}} 
\FlaTwoByOne
{ \widehat B_T }
{ \widehat B_B }
\]
or, equivalently,
\[
\FlaTwoByOne
{ B_T }
{ B_B }
=
\FlaTwoByOne
{ L_{TL} \widehat B_T }
{ L_{BL} \widehat B_T + L_{BR} \widehat B_B }
\]
so that, upon completion
\[
\FlaTwoByOneNoPar
{ B_T = L_{TL} \widehat B_T }
{ B_B = L_{BL} \widehat B_T + L_{BR} \widehat B_B }
\]
From this, we can choose two loop invariants:
\begin{description}
	\item
	{\bf Invariant 1:}
	$
	\FlaTwoByOne
	{ B_T = \widehat B_T }
	{ B_B = L_{BR} \widehat B_B }
	$. (The top part has been left alone and the bottom part has been partially computed).
	\item
	{\bf Invariant 2:}
	$
	\FlaTwoByOne
	{ B_T = \widehat B_T }
	{ B_B = L_{BL} \widehat B_T + L_{BR} \widehat B_B }
	$. (The top part has been left alone and the bottom part has been completely computed).
\end{description}

\subsection{PME 2}

To derive the second PME, partition
\[
B \rightarrow \FlaOneByTwo{ B_L }{ B_R }
\]
and does not partition $ L $.
Substituting these into the postcondition
yields
\[
\FlaOneByTwo{ B_L }{ B_R }
=
L \FlaOneByTwo{ \widehat B_L }{ \widehat B_R }
\]
or, equivalently,
\[
\FlaOneByTwo{ B_L }{ B_R }
=
\FlaOneByTwo{ L \widehat B_L }{ L \widehat B_R }
\]
so that, upon completion
\[
\FlaOneByTwoNoPar{B_L = L \widehat B_L}
{B_R = L \widehat B_R}
\]
From this, we can choose two more loop invariants:
\begin{description}
	\item
	{\bf Invariant 3:}
	$\FlaOneByTwo{B_L = L \widehat B_L}
	{B_R = \widehat B_R}$. (The left part has been completely finished and the right part has been left untouched).
	\item
	{\bf Invariant 4:}
	$\FlaOneByTwo{B_L = \widehat B_L}
	{B_R = L \widehat B_R}$. (The left part has been completely finished and the right part has been left untouched).
\end{description}

\subsection{Notes}

How do I decide to partition the matrices in the postcondition?

\begin{itemize}
	\item
	Pick a matrix (operand), any matrix.  
	\item 
	If that matrix has 
	\begin{itemize}
		\item 
	a triangular structure (in storage), then you want to either partition is into four quadrants, or not at all.  Symmetric matrices and triangular matrices have a triangular structure (in storage).
		\item
	no particular structure, then you partition it vertically (left-right), horizontally (top-bottom), or not at all.
	\end{itemize}
	\item
	Next, partition the other matrices similarly, but conformally (meaning the 
	resulting multiplications with the parts are legal).
\end{itemize}
Take our problem here:  $ B := L B $.
Start by partitioning $ L $ in to quadrants:
\[
B = 
\FlaTwoByTwo{ L_{TL} }{0}
{ L_{BL}}{ L_{BR} }
		\widehat B.
\]
Now, the way partitioned matrix multiplication works, this doesn't make sense:
\[
B = 
\begin{array}[t]{c}
\underbrace{
\FlaTwoByTwo{ L_{TL} }{0}
{ L_{BL}}{ L_{BR}}
		\widehat B.
	}\\
	\FlaTwoByOne
	{ L_{TL} \times \mbox{something} + 0 \times \mbox{something}}
	{ L_{BL} \times \mbox{something} + L_{BR} \times \mbox{something}}
	\end{array}.
\]
So, we need to also partition $ B $ into a top part and a bottom part:
\[
\FlaTwoByOne{ B_T}{B_B}
= 
\begin{array}[t]{c}
\underbrace{
	\FlaTwoByTwo{ L_{TL} }{0}
{ L_{BL}}{ L_{BR} }
		\FlaTwoByOne{ \widehat B_T}{\widehat B_B}.
	}\\
	\FlaTwoByOne
	{ L_{TL}  B_T + B_B}
	{ L_{BL}  B_T + L_{BR}  B_B}
	\end{array}.
\]

Alternatively, what if you don't partition $ L $?  You have to partition {\em something} so let's try partitioning $ B $:
\[
\FlaTwoByOne{ B_T}{B_B}
=
L 
\FlaTwoByOne{ \widehat B_T}{\widehat B_B}
\]
But that doesn't work...
Instead
\[
\FlaOneByTwo{ B_L}{B_R}
=
L 
\FlaOneByTwo{ \widehat B_L}{\widehat B_R}
=
\FlaOneByTwo{ L \widehat B_L}{ L \widehat B_R}
\]
works just fine.  